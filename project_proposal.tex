
%% Select document class - DO NOT CHANGE
\documentclass{sna}

%% Select any of standard LaTeX2e packages, e.g.
%  \usepackage{czech}
%  \usepackage{epsf}
\usepackage{graphicx}
\usepackage{amsmath}
\usepackage{esint}
%  \usepackage{amsfonts}
\usepackage{hyperref}
%\usepackage{subcaption}
\usepackage{subfig}
\usepackage{enumitem}

\usepackage{array}
\newcolumntype{L}[1]{>{\raggedright\let\newline\\\arraybackslash\hspace{0pt}}m{#1}}
\newcolumntype{C}[1]{>{\centering\let\newline\\\arraybackslash\hspace{0pt}}m{#1}}
\newcolumntype{R}[1]{>{\raggedleft\let\newline\\\arraybackslash\hspace{0pt}}m{#1}}

\def\vc#1{\mathbf{\boldsymbol{#1}}}     % vector
\def\abs#1{\left|#1\right|}
\def\norm#1{\| #1 \|}
\def\d{\mathrm{d}}

\newcommand{\bx}{\vc{x}}
\newcommand{\R}{\mathbf{R}}
\newcommand{\prob}[1]{Problem~{#1}}
\newcommand{\fig}[1]{\hyperref[#1]{Figure \ref{#1}}}

\begin{document}

%% Specify the paper: title, author(s), home institution and city
\info{Extended finite elements for mixed-hybrid model of Darcy flow}
     {P. Exner}
     {Project proposal}


\section{Abstract}
Research is aimed at modelling of groundwater flow on multi-dimensional meshes with finite element method. 
The general problems of these models (also in many other applications) is disability to capture small-scale 
phenomena in large domains. Instead of using adaptive meshing approach, extended finite element methods (XFEM) 
are very popular today. However, their usage with multi-dimensional discretization is still developing 
and has many opened questions. The main goal is to propose and possibly implement a proper XFEM in 
a model of groundwater flow for 1D-3D incompatible meshes.

\section{Introduction}
Mathematical modelling plays a very important role in science and also our daily lives throughout many different
fields of knowledge. Using modern finite element methods, we are able to simulate, investigate and predict
various phenomena of both nature and industrial character. Recent advances in the field of finite element methods
enable us to solve models of various scales, dimensions, to overcome numerical problems and also 
to couple interfering features together.

A large set of problems with finite element models, that people nowadays deal with, is connected with 
insufficient accuracy in cases where the model includes large and very small scale phenomena at the same time.
We can imagine a simulation of groundwater flow in a large domain (hundreds of metres) which can be significantly
influenced by thin fractures in the porous media or artificial wells and boreholes (several centimetres in diameter).
These disturbances bring discontinuities and singularities into the model and the finite elements are
unable to capture them accurately enough.

Adaptive meshes can be used in such cases, but it can cost a lot of computational power to build a very fine mesh,
and it also require very robust meshing techniques when complex geometries are in question.

Alternatively, multidimensional models are being developed. These decompose the geometry
in objects of different dimension (e.g. 2D fractures, 1D wells, 0D point sources), create meshes of the objects independently,
and then they must deal with coupling of the modelled processes among the dimensions. 

Next, there are the so called extended finite element methods (XFEM); or partition of unity methods (PUM) or generalized
finite element methods (GFEM), which names are sometimes used interchangeably. 
These enable us to take advantage of a~priori knowledge of the model solution character.
Then, we are able to incorporate non-polynomial functions, like jumps, into the finite element solution.
\newline{}

The proposed research is aimed at further development of the XFEM and its usage in multidimensional models. 
We intend to create a model with incompatible mesh, where elements of different dimensions intersect
arbitrarily, and then use the XFEM to glue the processes in different dimensions back together. This of course is a very ambitious
idea, therefore we narrow our plan to 1D-3D coupling. 
The proposed research is intended to be realized during a doctoral trainee-ship. 

Our research domain is groundwater modelling, therefore we apply the method on such model.
There are, of course, various other applications for such models apart from groundwater flow and transport in fractured porous media:
geothermal problematic and gathering geothermal energy; oil and gas extraction and also gas storage;
construction of safe underground nuclear waste deposits; investigation of transport of substances in human body
and more.


% Brezzi, Fortin - mixed hybrid method
% Jaffre - multidimensional models for flow


% Therefore, I kindly ask for funding of such traineeship to help me covering the expenses on accomodation, food
% and transportation in M{\" u}nich.

% mention definitely connection with Flow123d, importance for Barbara's team

\section{Personal interest}
This traineeship is a very important part of my Ph.D. study and my academic career in general.
The research plan is composed in the way that it continuously follows my previous work which was
aimed at the study and usage of the extended finite element methods in 2D problems with point singularities.
I will refer to my current research in Section \ref{sec:background}.
The results of the trainee-ship will make a significant part of my doctoral thesis.

To my best knowledge, there is nobody in the Czech republic who is intensively studying and developing extended finite element
methods in multi-dimensional problems at the moment. On the other hand, it is a hot topic worldwide and there
are many opened questions to be answered yet. I ensured myself about that during the conference X-DMS\footnote{X-DMS stands for eXtended 
Discretisation MethodS, Ferrara, 9.-11. September 2015, see \url{http://x-dms2015.sciencesconf.org/})}, where
quite a small umount of works connecting XFEM and multidimensional model of flow was presented. 
I will again write down most important references in Section \ref{sec:background}.

Further, I take part in a academic research project at my home university, that is focused on a development
of groundwater flow and transport simulator Flow123d\footnote{Flow123d web page: \url{http://flow123d.github.io/}}. 
The speciality of the software is usage of mixed hybrid method
for modelling groundwater flow on multi-dimensional meshes. I would like to transfer some of my experience
into this software which would support my thesis in its application part.

\section{Project Summary}
The traineeship will take place at the Technical University of M{\" u}nich, under the supervision of professor
Barbara Wolmuth. The proposed length of the trainee-ship is 6 months. The leading topic of the research is 
\textbf{extended finite elements for mixed-hybrid model of Darcy flow}. It is aimed at modelling of groundwater 
flow in fractured porous media on multi-dimensional meshes using the extended finite element methods. 
We can summarize the research plan into several items:
\begin{itemize}
  \item Research of current works concerning combination of XFEM and Mixed-hybrid method, in particular for Laplace equation. 
  \item Propose suitable enrichments for the pressure and the velocity that are stable (satisfy inf-sup condition). 
        Emphasise good approximation of the velocity field which is necessary for the transport equation. 
        Possible enrichments are: wells (point source in 2D, line in 3D), fractures (discontinuity in velocity), 
        boundary of 1D and 2D  fractures. To narrow the plan we will concentrate on enrichment for 3D-1D interaction.
    %   The aim is to have working enrichment for 3D-1D interaction and some intuition about other useful enrichments that we want implement in future.
  \item Implement the enrichment for 1D-3D (wells in 3D domain) in Flow123d.
\end{itemize}

% place, length, title, topic, aim, results


\section{Detailed project descrition}
\subsection{Background} \label{sec:background}




Regarding my research, I~have been working on the following since I~started my Ph.D. study. 
I~have developed a~model for quasi-3D aquifers-wells problem according to Gracie and Craig~\cite{gracie_modelling_2010,craig_using_2011}.
The main idea was usage of the XFEM for dealing with the singularity in pressure at the wells (0D-2D coupling).
I~have implemented different finite element methods: standard FEM with adaptive refinement, 
standard XFEM and corrected XFEM~\cite{fries_corrected_2008,fries_xfem_overview_2010},
stable generalized finite element methods (SGFEM) \cite{babuska_stable_2012, gupta_stable_2013}.
I~have measured their convergence, investigated their properties and compared them. 
I~also improved adaptive integration on enriched elements which resulted in optimal convergence rates of
all mentioned extended finite element methods (the original integration in \cite{gracie_modelling_2010,craig_using_2011}
is not that robust and accurate enough). 
The implementation of the compared methods has been done in C++ language using the Deal II~\cite{bangerth_deal.ii_2007}, 
the finite element library not supporting any enrichment techniques at the moment.
The results are summarized in the article~\cite{exner_2015}, which is being published at the moment.


In general, we are interested in the model of groundwater flow in fractured porous media with combined mesh dimensions.
This model can be discretized using the mixed-hybrid finite element method, which is described in
well known book by Brezzi and Fortin~\cite{brezzi_mixed_1991}. 
The actual application of the mixed-hybrid method in this model is studied in several articles.
In \cite{brezina_mixed-hybrid_2010}, the weak formulation and its discretization using Raviart-Thomas finite 
elements is written. The linear system is then reduced using the Schur complement~\cite{maryska_mixed-hybrid_1995}
and solved.
In \cite{brezina_2012} a mortar-like method is used to deal with discrete coupling between equations on incompatible 1D-2D meshes 
of different dimensions. The drawback of this approach is that it uses continuous approximation of velocity, so
it cannot approximate the discontinuity of the velocity near the fractures accurately enough.
In \cite{sistek_bddc_2015}, the theoretic part includes the weak and the discrete formulation and also shows
the uniqueness of the discrete solution. Further, the authors discuss application and results of BDDC 
(Balancing Domain Decomposition by Constraints) method used for solution of the linear system.

Many researchers are using some kind of extended finite element method. The applications are mainly aimed
at problems in mechanics, much less people are working with XFEM in field of flow modelling. One of the reasons is
that the stability of the discrete mixed form is more sensitive on the choice of the finite elements used 
in flow problems.



TUM - experience in underground water flow modelling, multi-dimensional models...

(PUM, pum in mechanics, Zuzino - 1d-3d, mixed hybrid method)
what we have done (0d-2d model, comparison of pum, adaptive integration, Flow123d - intersections 1d-3d 2d-3d)

We want to investigate what kind of enrichment in XFEM is suitable for the mixed-hybrid model. The question
is whether we are able to find such enrichment that would not harm the satisfaction of the discrete inf-sup
condition.



\section{Research schedule}
\begin{center}

  \begin{tabular}{R{3cm}|L{12cm}}
    \hline
    length(days) / date & activity \\ \hline \hline
    20.6.2016 & arrival \\ \hline
    30 & research on current works concerning combination XFEM and Mixed-hybrid method \\
    60 & propositions for enrichment of pressure and velocity   \\
    90 & (implementation) prepare XFEM support in Flow123d  \\
    30 & (implementation) prepare intersection lookup module    \\
    30 & (implementation) prepare FE interface and its support for enrichment   \\
    30 & (implementation) prepare DoFHandler for enriched degrees of freedom    \\  
    30 & (implementation) define algebraic system with enriched DoFs  \\
    20.12.2016 & departure \\ \hline
  \end{tabular}
\end{center}

% \bibliographystyle{abbrv}
\bibliographystyle{elsarticle-num} 
\bibliography{citace}
    
\end{document}
